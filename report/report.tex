\documentclass[a4paper,12pt]{article}
\usepackage{graphicx}
\usepackage{amsmath}
\usepackage{hyperref}
\usepackage{geometry}
\geometry{margin=1in}

\title{Data Mining Report}
\author{Your Name}
\date{\today}

\begin{document}

\maketitle

\begin{abstract}
This report details and highlights the work performed in the context of data mining. It includes motivations, thorough analysis, observations, and limitations for each task.
\end{abstract}

\section{Introduction}
\label{sec:introduction}
Provide an overview of the report, including the main objectives and structure.

\section{General Guidelines}
\label{sec:guidelines}
This section provides a guide on the expected report's contents. Please keep in mind that the report has the goal of detailing and highlighting your work. As such, all analyses must contain:
\begin{itemize}
    \item \textbf{Motivations}: Why did you perform this analysis, rather than another one? Why did you look to create one feature/representation, rather than another?
    \item \textbf{Thorough analysis}: Each analysis should be performed in a reasonably large set of settings, e.g., considering several hyperparameters for the algorithms you run, and, when appropriate, choosing a set of hyperparameters. Please justify your choices.
    \item \textbf{Observations}: What insight and/or information did you gain from each analysis you performed?
    \item \textbf{Limitations}: How strong are the analytical results, and observations you have found?
\end{itemize}

\section{Task 1: Data Understanding}
\label{sec:task1}
\subsection{Assessing Data Quality}
Analyze the dataset, including assessing data quality.

\subsection{Data Distribution}
Analyze the data distribution.

\subsection{Relationships Between Features}
Analyze the relationships between features.

\section{Task 2: Data Transformation}
\label{sec:task2}
The data transformation task includes three subtasks:

\subsection{Feature Engineering and/or Novel Feature Definition}
Improve the quality of your data by tackling eventually missing/incorrect values, either engineering or defining novel features of interest.

\subsection{Outlier Detection}
Detect and handle outliers in the dataset.

\subsection{Revamped Data Understanding}
A revamped data understanding task, now including the features of point 1, and eventual considerations of point 2.

\section{Task 3: Clustering}
\label{sec:task3}
Leverage clustering algorithms to identify and describe the groups of instances you have found. The section should consider all clustering algorithms tackled in the course:
\begin{itemize}
    \item k-means clustering
    \item Density-based clustering
    \item Hierarchical clustering
\end{itemize}

\subsection{Final Observations and Comparisons}
Present final observations and comparisons on different clusterings. Additionally, the group can experiment with additional clustering algorithms available here. Additional algorithms can yield up to 2 bonus points in evaluation.

\section{Conclusion}
\label{sec:conclusion}
Summarize the main findings and insights from the report.

\end{document}